\documentclass[a4paper, 12pt]{article}
\usepackage[UTF8]{ctex}
\usepackage{graphicx}
\graphicspath{{./images/}}

\begin{document}
\title{\textbf{系统开发工具基础第一周实验报告}}
\author{袁庆康 23020007154 \\Git:https://github.com/kangya87/Xi\_Ton}
\date{\today}
\maketitle
\section{\large Git的练习内容、结果、感悟}
\subsection{\small 创建GitHub仓库}
打开VS创建一个新的项目,选择创建GitHub仓库

\includegraphics[width=6cm, height=6cm]{Pict2}
\includegraphics[width=6cm, height=6cm]{Pict1}

创建后,相应的GitHub账号种便有了该仓库.
\subsection{\small 文件上传}
\includegraphics[width=6cm, height=6cm]{Pict3}
\includegraphics[width=6cm, height=6cm]{Pict4}

打开VS,进行部分修改后将修改保存上传。

打开GitHub桌面端,将该仓库加入到桌面端中。

\includegraphics[width=12cm, height=6cm]{Pict5}

在桌面端中同意并上传修改。

\includegraphics[width=10cm, height=5cm]{Pict6}

至此,完成了基本的库的创立,此后修改之类的操作相同。
\subsection{\small 创建小组}
\includegraphics[width=12cm, height=6cm]{Pict}
创建小组后,可以导入已有的仓库供小组成员共同开发使用,也可以通过邮箱邀请他人加入小组,小组内的成员可以对仓库中的文件进行修改并且保存上传,经管理员同意后完成上传。
\subsection{\small 感悟}
GitHub极大的简化进行大型项目时,多人开发项目的操作流程,使得项目开发更加有序便利。

\section{\large Latex的练习内容、结果、感悟}
\subsection{\small 基础编写技能}
\includegraphics[width=12cm, height=6cm]{Pict8}

根据Latex入门,学到基本的标题、章节、标签、注释、字体等内容。实例如下:  

\includegraphics[width=12cm, height=6cm]{Pict9}
\subsection{\small 模板下载和修改}
\includegraphics[width=12cm, height=6cm]{Pict11}

从网络中下载Latex模板后,根据需要对其文字进行一定的增添或减少,就可以方便的得到属于自己的Latex文件。
\subsection{\small 感悟}
Latex可以帮助我们快速方便的做出美观、简洁的文档。使得报告、论文、简介等的编写更加方便。


\end{document}