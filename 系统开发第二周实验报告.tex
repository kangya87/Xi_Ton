\documentclass[a4paper, 12pt]{article}
\usepackage[UTF8]{ctex}
\usepackage{graphicx}
\graphicspath{{./images/}}

\begin{document}
\title{\textbf{系统开发工具基础第二周实验报告}}
\author{袁庆康 23020007154 \\Git:https://github.com/kangya87/Xi\_Ton}
\date{\today}
\maketitle

\pagenumbering{roman}
\tableofcontents
\newpage
\pagenumbering{arabic}

\section{\large Shell的练习内容、结果、感悟}
\subsection{\small Shell文件的定位和浏览}

\includegraphics[width=12cm, height=6cm]{Pict1}

利用cd+文件名(该文件夹下的),可以使定位移动到该文件;cd+..将定位移动到上一文件。

输入ls 可以浏览当前文件中的所有文件。
\subsection{\small 创建浏览文件}
\includegraphics[width=12cm, height=2.5cm]{Pict2}

输入\textbf{echo 任意语句} \textgreater \textbf{文件名.文件格式} 可以在当前定位下创建该文件并输入该语句。

输入\textbf{cat 文件名.文件格式} 可以打印输出该文件的内容。
输入\textbf{cat 文件名.文件格式} \textgreater \textbf{文件名1.文件格式} 可以将文件中的内容传到文件1中。
\subsection{\small 打开文件编译模式(重加载)}
\includegraphics[width=12cm, height=2.7cm]{Pict3}

创建文件后输入\textbf{cat>test.txt(例)}可以清空该文件,并且写入内容。
\subsection{\small 打开文件编译模式(后缀)}
\includegraphics[width=12cm, height=5cm]{Pict4}

创建文件后输入\textbf{cat>test.txt(例)}可以在原先文件的基础上写入新的内容
\subsection{\small 创建文件夹}
\includegraphics[width=12cm, height=0.7cm]{Pict5}

输入\textbf{mkdir+文件名}可以在当前定位下创建文件夹。
\subsection{\small 文件权限查询}
\includegraphics[width=12cm, height=2cm]{Pict6}

\textbf{ls -l}可以显示当前定位下文件内容,修改事件拥有者和管理权限等内容。  

\subsection{\small 文件权限修改}
\includegraphics[width=12cm, height=2cm]{Pict7}

输入\textbf{chmod 777(最高权限) 文件}可以修改权限
7-rwx 6-rw 5-rx 4-r 3-wx 2-w 1-x 0-无
三位数分别是属主、属组、其它用户 
\subsection{\small bash文件的创建}
\includegraphics[width=12cm, height=2.2cm]{Pict8}

bash文件的后缀是sh如,\textbf{test.sh}
\subsection{\small bash中连接运算符的规律}
\includegraphics[width=12cm, height=5cm]{Pict9}

bash中连接运算符和c、c++中一样,\textbackslash \textbackslash 为或,\&\&为与.
\subsection{\small bash文件的运行}
\includegraphics[width=12cm, height=3cm]{Pict10}

输入\textbf{bash -v test.sh}可以运行并打印该文件运行结果.
\subsection{\small 感悟}
Shell相当于是一个翻译,把我们在计算机上的操作或我们的命令,翻译为计算机可识别的二进制命令,传递给内核,以便调用计算机硬件执行相关的操作;同时,计算机执行完命令后,再通过Shell翻译成自然语言,呈现在我们面前。
\section{\large vim的练习、结果、感悟}
\subsection{\small vim环境的调试}
\includegraphics[width=12cm, height=6cm]{Pict11}

在终端中输入wsl可以使编译器模拟Linux虚拟机环境,在此环境可以用vim编译
\subsection{\small vim编译器的打开}
\includegraphics[width=12cm, height=6cm]{Pict12}

输入vim便可以进入vim编译模式
\subsection{\small vim编译模式的改变}
\includegraphics[width=12cm, height=6cm]{Pict13}

点击i键可以进入vim插入模式;r进入替换模式;v进入一般化模式;按下<ESC>退回正常模式.
\subsection{\small vim保存内容到文件}
\includegraphics[width=12cm, height=6cm]{Pict14}

Esc返还正常模式后,\textbf{输入:w 文件名},将内容保存到该文件
\subsection{\small vim读取文件内容}
\includegraphics[width=12cm, height=6cm]{Pict15}

输入\textbf{:e 文件名}读取并打印该文件.
\subsection{\small 感悟}
Vim 是一个 多模态 编辑 器:它对于插入文字和操纵文字有不同的模式。Vim 是可编程的(可以使用 Vimscript 或者像Python一样的其他程序语言),Vim 的接口本身也是一个程序语言:键入操作(以及其助记名) 是命令,这些命令也是可组合的。
\end{document}