\documentclass[a4paper, 12pt]{article}
\usepackage[UTF8]{ctex}
\usepackage{graphicx}
\graphicspath{{./images/}}

\begin{document}
\title{\textbf{系统开发工具基础第三周实验报告}}
\author{袁庆康 23020007154 \\Git:https://github.com/kangya87/Xi\_Ton}
\date{\today}
\maketitle

\pagenumbering{roman}
\tableofcontents
\newpage
\pagenumbering{arabic}

\section{\large Python的练习内容、结果、感悟}
\subsection{\small Python的输入、输出}

\includegraphics[width=12cm, height=3.5cm]{Pict1}

python的输入是\textbf{input("提示内容")}

输出是\textbf{print()}
\subsection{\small 字符串切片}

\includegraphics[width=12cm, height=2cm]{Pict2}

python中的字符串有类似c++中的string功能,python的string相比c++有更多丰富的功能,比如切片功能。


\subsection{\small for、while用法}

\includegraphics[width=12cm, height=5cm]{Pict3}

除了细微的语法区别,python中的for和while有着和c++基本相同的功能.
\subsection{\small 函数的用法}

\includegraphics[width=12cm, height=4cm]{Pict4}

python中函数的构建统一用def做注释,并且可以不给参数定义类型。
\subsection{\small 列表的用法}

\includegraphics[width=12cm, height=4cm]{Pict5}

python中列表的代表是list,使用\textbf{lst=list("字符串")},可以将字符串分割存储到lst中。
\subsection{\small 元组的用法}
\includegraphics[width=12cm, height=4cm]{Pict6}

元组在python中的代表是tuple,使用\textbf{t=tuple("字符串")},可以将字符串分割存储到t中。与列表不同是,元组中的元素是不可以修改的。

\subsection{\small 感悟}

python是一个及其方便且功能强大的语言,不同于c++python将程序员从繁杂的代码中解救出来,更加快速制作繁杂庞大的项目。

但是python也有其弱项,因其庞大的库,它会使代码运行速度变得更加缓慢。
\section{\large Python视觉的练习、结果、感悟}
\subsection{\small PIL图片的打印}
\includegraphics[width=12cm, height=8cm]{Pic7}

使用\textbf{pil=Image.open("图片名")}可以打开该图片并存储到pil中,然后用\textbf{pil}打印该图片
\subsection{\small PIL的图片的存储}
\includegraphics[width=12cm, height=0.9cm]{Pict8}

\includegraphics[width=12cm, height=0.5cm]{Pict9}

使用\textbf{pil.save("新的图片名")}可以将新的图片存储到python的主文件夹里。
\subsection{\small PIL对图片的操作}
\includegraphics[width=12cm, height=8cm]{Pict10}

使用\textbf{pil.transpose(Imagge.FLIP\_LEFT\_RIGHT)}可以使图片发生镜面倒转,使用\textbf{pil.rotate(角度)}可以使图片发生该角度的转发。
\subsection{\small PIL画点画线}
\includegraphics[width=12cm, height=6cm]{Pict11}

用x,y来确定点的位置,然后使用\textbf{plot(x,y,"go-")}可以使用绿色线连接x和y的位置,同时使用\textbf{title("内容")}可以为图片打印标题。
\subsection{\small PIL切割图片}
\includegraphics[width=12cm, height=3.5cm]{Pict12}

使用\textbf{box(lx,uy,rx,dy)}可以确定由lx到rx,uy到dy的方形区域,然后使用\textbf{pil.crop(box)}可切割box确定区域的图片。
\subsection{\small PIL图片互动功能}
\includegraphics[width=12cm, height=6cm]{Pict14}

可以使用\textbf{ginput(3)}获取三次点击图片获取的位置信息,然后将位置信息存储后用plot函数进行画点、画图操作。
\subsection{\small PIL获取图片数组}
\includegraphics[width=12cm, height=3cm]{Pict13}

使用\textbf{im=array("图片名")}可以使im获取该图片的特定数组。
\subsection{\small PIL图片灰值的调整}
\includegraphics[width=12cm, height=8cm]{Pict15}

使用\textbf{array}获取图片数组后,对数组进行算法调控,来调整图片的灰值。
\subsection{\small PIL获取图像轮廓和直方图}
\includegraphics[width=12cm, height=9cm]{Pict16}

对获取的图像数组使用\textbf{hist(im.flatten(),128)}可以获取图片的直方图。
\subsection{\small 直方图均衡化}
\includegraphics[width=12cm, height=9cm]{Pict17}

可以使用图中hitteq算法对图片中色块出现频率和分布进行均衡化,然后打印出分别度更高的图片。
\subsection{\small 图像导数}
\includegraphics[width=12cm, height=6cm]{Pict18}

使用 scipy.ndimage.filters 模块的标准卷积操作来简单地实现。
\subsection{\small 模糊处理}
\includegraphics[width=12cm, height=8cm]{Pict19}

使用\textbf{gaussian\_filter(im,5)}可以对图像数组进行模糊化处理,再使用\textbf{Image.fromarray(im)}对数组进行反译,形成新的图像。
\subsection{\small 感悟}
本书中,使用广义的计算机视觉概念,包括图像扭曲、降噪和增强现实等。

计算机视觉是一门对图像中信息进行自动提取的学科。信息的内容相当广泛,包括三维模型、照相机位置、目标检测与识别,以及图像内容的分组与搜索等。
\end{document}